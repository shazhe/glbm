\documentclass{standalone}

\usepackage[english]{babel}
\usepackage[T1]{fontenc}
\usepackage[ansinew]{inputenc}

\usepackage{lmodern}	% font definition
\usepackage{amsmath}	% math fonts
\usepackage{amsthm}
\usepackage{amsfonts}

\usepackage{tikz}
\usetikzlibrary{decorations.pathmorphing} % noisy shapes
\usetikzlibrary{fit}					% fitting shapes to coordinates
\usetikzlibrary{backgrounds}	% drawing the background after the foreground

\begin{document}


% The state vector is represented by a blue circle.
% "minimum size" makes sure all circles have the same size
% independently of their contents.
\tikzstyle{process}=[circle,
                                    thick,
                                    minimum size=1.2cm,
                                    draw=blue!80,
                                    fill=blue!20]

% The measurement vector is represented by an orange circle.
\tikzstyle{data}=[circle,
                                                thick,
                                                minimum size=1.2cm,
                                                draw=orange!80,
                                                fill=orange!25]

% The control input vector is represented by a purple circle.
\tikzstyle{process}=[circle,
                                    thick,
                                    minimum size=1.2cm,
                                    draw=purple!80,
                                    fill=purple!20]

% The input, state transition, and measurement matrices
% are represented by gray squares.
% They have a smaller minimal size for aesthetic reasons.
\tikzstyle{matrx}=[rectangle,
                                    thick,
                                    minimum size=1cm,
                                    draw=gray!80,
                                    fill=gray!20]

% The system and measurement noise are represented by yellow
% circles with a "noisy" uneven circumference.
% This requires the TikZ library "decorations.pathmorphing".
\tikzstyle{noise}=[circle,
                                    thick,
                                    minimum size=1.2cm,
                                    draw=yellow!85!black,
                                    fill=yellow!40,
                                    decorate,
                                    decoration={random steps,
                                                            segment length=2pt,
                                                            amplitude=2pt}]

% Everything is drawn on underlying gray rectangles with
% rounded corners.
\tikzstyle{background}=[rectangle,
                                                fill=gray!10,
                                                inner sep=0.2cm,
                                                rounded corners=5mm]

\begin{tikzpicture}[>=latex,text height=1.5ex,text depth=0.25ex]
    % "text height" and "text depth" are required to vertically
    % align the labels with and without indices.
  
  % The various elements are conveniently placed using a matrix:
  \matrix[row sep=0.5cm,column sep=0.5cm] {
    % First line: Data
    &
        \node (u_1) [data]{Altimetry}; &        
        & 
        &       
        \node (u_3) [data]{GRACE}; &
        &
        \node (u_4) {}; &
        \\
        % Second line: measurement error & A matrix
        \node (w_1) [noise] {$\mathbf{w}_{k-1}$}; &
        \node (B_1) [matrx] {$\mathbf{B}$};       &
        \node (w_2)   [noise] {$\mathbf{w}_k$};     &
        \node (B_2)   [matrx] {$\mathbf{B}$};       &
        \node (w_3) [noise] {$\mathbf{w}_{k+1}$}; &
        \node (B_3) [matrx] {$\mathbf{B}$};       &
        \node (w_4) [noise] {$\mathbf{w}_{k+1}$}; &
        \node (B_4) [matrx] {$\mathbf{B}$};       &
        \\
        % Third line: the process and constraints
      
        \node (x_1) [process] {$\mathbf{x}_{k-1}$}; &
        &
        \node (x_2)   [process] {$\mathbf{x}_k$};     &
        &
        \node (x_3) [process] {$\mathbf{x}_{k+1}$}; &
        &
         \node (x_4) [process] {$\mathbf{x}_{k+1}$}; &
           \\
        % Fourth line: priors
        \node (v1) [noise] {$\mathbf{v}_{k-1}$}; &
        &
        \node (v2)   [noise] {$\mathbf{v}_k$};     &
        &
        \node (v3) [noise] {$\mathbf{v}_{k+1}$}; &
        &
        \node (v4) [noise] {$\mathbf{v}_{k+1}$}; &
        \\
    };
    
    % The diagram elements are now connected through arrows:
    \path[->]
        	% The main path between the
        
        (x_k-1) edge (H_k-1)				% Output path x -> H -> z
       
        (x_k)   edge (H_k)
       
        (x_k+1) edge (H_k+1)
       
        
       
        
        (w_k-1) edge (x_k-1)				% System noise w -> x
        (w_k)   edge (x_k)
        (w_k+1) edge (x_k+1)
        
        (u_k-1) edge (B_k-1)				% Input path u -> B -> x
        (B_k-1) edge (x_k-1)
        (u_k)   edge (B_k)
        (B_k)   edge (x_k)
        (u_k+1) edge (B_k+1)
        (B_k+1) edge (x_k+1)
        ;
    
    % Now that the diagram has been drawn, background rectangles
    % can be fitted to its elements. This requires the TikZ
    % libraries "fit" and "background".
    % Control input and measurement are labeled. These labels have
    % not been translated to English as "Measurement" instead of
    % "Messung" would not look good due to it being too long a word.
    \begin{pgfonlayer}{background}
        \node [background,
                    fit=(u_k-1) (u_k+1),
                    label=left:Data:] {};
        \node [background,
                    fit=(x_k-1) (x_k+1),
                    label = left:Processes:] {};
        \node [background,
                    fit=(v_k-1) (H_k+1),
                    label=left:Priors:] {};
    \end{pgfonlayer}
\end{tikzpicture}



\end{document}